\documentclass{rapport}
\usepackage{subcaption}
\usepackage{adjustbox}
\usepackage{lipsum}
\usepackage{gensymb}
\usepackage{float}
\usepackage{wrapfig}
\usepackage{amsmath} % Pour les environnements mathématiques
\usepackage{amssymb} % Pour les symboles supplémentaires comme R (réels)
\usepackage{graphicx} % Required for inserting images
\title{BE Automatique} %title of the file

\begin{document}

%----------- Report information ---------

\logo{logos/logo_n7.png}
\uni{\textbf{ENSEEIHT}}
\ttitle{Liste des sujets - Projet long TOB} %title of the file
\subject{Sujets} % Subject name
\topic{Projet Long - TOB} % Topic name

\professor{G. \textsc{Dupont}} % information related to the professor

\students{Alexandre \textsc{Lescot}\\
          Camille \textsc{Meyer}\\
          Doryan \textsc{Benoit}\\
          Iman-Norr \textsc{Draou}\\
          Mathier \textsc{Trahand}\\
          Oscar \textsc{Mautin}\\
          Sophie \textsc{Girardot}\\
          Yacine  \textsc{Meziani}} % information related to the students

%----------- Init -------------------
        
\buildmargins % display margins
\buildcover % create the front cover of the document
\toc % creates the table of contents

%------------ Report body ----------------

\section{Liste des projets par préférence}

\hypertarget{simcity}{%
\subsection{Simcity}\label{simcity}}

Simulation 2D d'une ville intelligente, vue du dessus. Contient la
simulation des individus qui y évoluent ainsi que leurs interactions.
Simulation du trafic routier, des comportements dans les carrefours,
feux rouges, etc\ldots{}

\hypertarget{logiciel-de-montage-viduxe9o}{%
\subsection{Logiciel de montage vidéo}\label{logiciel-de-montage-viduxe9o}}

Logiciel de montage vidéo logiciel de montage vidéo simple et intuitif
prenant en charge les fonctionnalités de base :
\begin{itemize}
    \item Compatibilité multimédia : prise en charge des formats courants de vidéo, audio et image. 
    \item Timeline interactive : déplacement et superposition des clips avec la possibilité de couper. 
    \item Compositing de base : ajustement et transformation des clips (zoom, translation, opacité).
\end{itemize}

On garde la possibilité d'ajouter des fonctionnalités avancées si le temps le permet (ex. fond vert).

\hypertarget{moteur-de-voxels-et-guxe9nuxe9ration-procuxe9durale}{%
\subsection{Moteur de voxels et génération procédurale}\label{moteur-de-voxels-et-guxe9nuxe9ration-procuxe9durale}}

Clone du jeu Minecraft. Au début, on se concentre uniquement sur la
création du moteur 3D permettant d'afficher des voxels (pixels en 3
dimensions) puis, si le temps le permet, ajout de la possibilité de
générer le terrain de manière procédurale.

\hypertarget{paint-logiciel-de-dessin-et-de-montage-photo-sur-ordinateur}{%
\subsection{Paint (logiciel de dessin et de montage photo sur
ordinateur)}\label{paint-logiciel-de-dessin-et-de-montage-photo-sur-ordinateur}}

Logiciel permettant à l'utilisateur de dessiner sur une image, puis ajout d'autres fonctionnalités si le temps le permet. Les fonctionnalités (liste non exhaustive) : 
\begin{itemize}
    \item Ajout d'une logique de calques
    \item Application de masques
    \item Selection de pixels similaires
\end{itemize}

\hypertarget{guxe9nuxe9rateur-de-colloscope}{%
\subsection{Générateur de colloscope}\label{guxe9nuxe9rateur-de-colloscope}}

Créer une application pour générer des programmes de colles en prépa.

L'utilisateur renseigne les groupes de colles, les colleurs, leurs
disponibilités et les emplois du temps des élèves (selon les options).
Ensuite, il définit le nombre de colles par semaine et la fréquence dans
chaque matière.

L'algorithme génère un emploi du temps optimisé qui respecte au mieux
les contraintes et fait varier les colleurs. Une version PDF du planning
est téléchargeable pour un partage facile.

\end{document}
