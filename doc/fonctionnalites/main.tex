\documentclass{rapport}
\usepackage{subcaption}
\usepackage{adjustbox}
\usepackage{lipsum}
\usepackage{gensymb}
\usepackage{float}
\usepackage{wrapfig}
\usepackage{amsmath} % Pour les environnements mathématiques
\usepackage{amssymb} % Pour les symboles supplémentaires comme R (réels)
\usepackage{graphicx} % Required for inserting images
\title{BE Automatique} %title of the file

\begin{document}

%----------- Report information ---------

\logo{logos/logo_n7.png}
\uni{\textbf{ENSEEIHT}}
\ttitle{Fonctionnalités - Projet long TOB\\Logiciel de montage vidéo} %title of the file
\subject{Fonctionnalités du logiciel de montage} % Subject name
\topic{Projet Long - TOB} % Topic name

\professor{G. \textsc{Dupont}} % information related to the professor

\students{Alexandre \textsc{Lescot}\\
          Camille \textsc{Meyer}\\
          Doryan \textsc{Benoit}\\
          Iman-Norr \textsc{Draou}\\
          Mathier \textsc{Trahand}\\
          Oscar \textsc{Mautin}\\
          Sophie \textsc{Girardot}\\
          Yacine  \textsc{Meziani}} % information related to the students

%----------- Init -------------------
        
\buildmargins % display margins
\buildcover % create the front cover of the document
\toc % creates the table of contents

%------------ Report body ----------------

\section{MVP}
\begin{itemize}
    \item Timeline
    \item Preview
    \item Import d'un type de vidéo
    \item Voir les fichiers importés
    \item Mettre les vidéo dans une timeline
    \item Voir, entendre et manipuler au moins 1 type de fichier vidéo+audio 
\end{itemize}

\section{Étape intermédiaire}
\begin{itemize}
    \item Couper une vidéo
    \item Coller deux vidéos
    \item Ecouter et manipuler au moins 1 type de fichier audio
    \item Exporter la timeline 
\end{itemize}



\section{Fonctionnalités supplémentaires}
\begin{itemize}
    \item Ajout de texte
\end{itemize}

\end{document}
